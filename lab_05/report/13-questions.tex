\section{ВОПРОСЫ}

\begin{enumerate}
    \item \textbf{В каком фрагменте программы сформулировано знание? Это знание о чем на формальном уровне?}

        Знания о предметной области выражаются на языке Пролог в виде предложений, называемых утверждениями (clauses). Каждое утверждение заканчивается точкой и описывает какое-либо отношение, свойство, объект или закономерность. Структура утверждения проста и имеет одну из форм:
        \begin{itemize}
            \item <заголовок>. -- факт
            \item <заголовок>:- <тело>. -- правило, где заголовок является предикатом и полностью характеризует описываемое отношение.
        \end{itemize}

    \item \textbf{Что содержит тело правила?}

        Тело правила содержит условие истинности для этого правила.

    \item \textbf{Что дает использование переменных при формулировании знаний? В чем отличие формулировки знания с помощью термов с одинаковой арностью при использовании одной переменной и при использовании нескольких переменных?}

        Множество фактов образуют простейшую программу Пролога. Но атомарный предикат факта может содержать переменныев качестве аргументов или неосновные термы. В этом случае по умолчанию считается, что добавлен квантор всеобщности $\forall$ с переменными предиката. Такие факты называются универсальными: они истинны для любых значений переменных. Например, любит(X, яблоко) $\leftarrow$ означает, что любой объект программы ``любит яблоко''. Универсальные факты сокращают запись программы.

    \item \textbf{С каким квантором переменные входят в правило, в каких пределах переменная уникальна?}

        На все переменные в имени предиката наложен квантор всеобщности $\forall$, на переменные в теле предиката, которые отстутствуют в имени, наложен квантор существования $\exists$.

        Областью действия переменной в Prolog является одно предложение. В разных предложениях может использоваться одно имя перменной для обозначения разных объектов. Исключением является анонимная переменная. Каждая анонимная переменная -- это отдельный объект.

    \item \textbf{Какова семантика (смысл) предложений раздела DOMAINS?  Когда, где и с какой целью используется это описание?}

        Предложения раздела DOMAINS описывают новые типы данных, которые используются в разделах PREDICATES, для описания отношений, в которых можно будет использовать более частные случаи типов данных.

    \item \textbf{Какова семантика (смысл) предложений раздела PREDICATES? Когда, и где используется это описание? С какой целью?}

        Предикаты используются для описания фактов в разделе CLAUSES. Предикаты описывают какие домены будет принимать то или иное правило. Это описание используется для создания договоренности о том, в каком порядке будут идти термы в отношениях.

    \item \textbf{Унификация каких термов запускается на самом первом шаге работы системы? Каковы назначение и результат использования алгоритма унификации? }

        Первым запускается унификация термов вопроса.

        Для выполнения логического вывода используется механизм (алгоритм) унификации, встроенный в систему. Унификация – операция, которая позволяет формализовать процесс логического вывода (наряду с правилом резолюции). С практической точки зрения - это основной вычислительный шаг, с помощью которого происходит:
        \begin{itemize}
            \item Двунаправленная передача параметров процедурам,
            \item Неразрушающее присваивание,
            \item Проверка условий (доказательство).
        \end{itemize}
        В процессе работы система выполняет большое число унификаций.

    \item \textbf{В каком случае запускается механизм отката?}

        В том месте программы, где возможен выбор нескольких вариантов, Пролог сохраняет в специальный стек точку возврата для последующего возвращения в эту позицию. Точка возврата содержит информацию, необходимую для возобновления процедуры при откате. Выбирается один из возможных вариантов, после чего продолжается выполнение программы.

Во всех точках программы, где существуют альтернативы, в стек заносятся указатели. Если впоследствии окажется, что выбранный вариант не приводит к успеху, то осуществляется откат к последней из имеющихся в стеке точек программы, где был выбран один из альтернативных вариантов. Выбирается очередной вариант, программа продолжает свою работу. Если все варианты в точке уже были использованы, то регистрируется неудачное завершение и осуществляется переход на предыдущую точку возврата, если такая есть. При откате все связанные переменные, которые были означены после этой точки, опять освобождаются.
\end{enumerate}
