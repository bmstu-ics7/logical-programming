\section{ФОРМИРОВАНИЕ ОТВЕТА}

Для одного из вариантов ВОПРОСА и каждого варианта задания 2 составить таблицу, отражающую конкретный порядок работы системы: Т.к. резольвента хранится в виде стека, то состояние резольвенты требуется отображать в столбик: вершина – сверху! Новый шаг надо начинать с нового состояния резольвенты!

{
\small
\begin{longtable}{|p{1.15cm}|p{4cm}|p{6cm}|p{6cm}|}
    \caption{maxWithCut(1, 5, 3, Max)} \\
    \hline
    № шага & Состояние резольвенты, и вывод: дальнейшие действия (почему?) & Для каких термов запускается алгоритм унификации: T1=T2 и каков результат (и подстановка) & дальнейшие действия: прямой ход или откат (почему и к чему приводит?) \\
    \hline
    1 & maxWithCut(1, 5, 3, Max) & Подстановка: A = 1, B = 5, C = 3, Max = Max & Прямой ход \\
      & & maxWithCut(1, 5, 3, Max) & \\
      & & maxWithCut(A, B, C, Max) & \\
    \hline
    2 & A > B & Проверка: 1 > 5 & Обратный ход \\
      & A > C & & \\
      & Max = A & & \\
      & ! & & \\
    \hline
    3 & maxWithCut(1, 5, 3, Max) & Подстановка: B = 5, C = 3, Max = Max & Прямой ход \\
      & & maxWithCut(1, 5, 3, Max) & \\
      & & maxWithCut(\_, B, C, Max) & \\
    \hline
    4 & B > C & Проверка: 5 > 3 & Прямой ход \\
      & Max = B & & \\
      & ! & & \\
    \hline
    5 & Max = B & Подстановка: Max = 5 & Прямой ход \\
      & ! & & \\
    \hline
    6 & ! & \textbf{Результат:} Max = 5 & Обратный ход \\
    \hline
\end{longtable}
}

{
\small
\begin{longtable}{|p{1.15cm}|p{4cm}|p{6cm}|p{6cm}|}
    \caption{maxWoutCut(3, 2, 1, Max)} \\
    \hline
    № шага & Состояние резольвенты, и вывод: дальнейшие действия (почему?) & Для каких термов запускается алгоритм унификации: T1=T2 и каков результат (и подстановка) & дальнейшие действия: прямой ход или откат (почему и к чему приводит?) \\
    \hline
    1 & maxWoutCut(3, 2, 1, Max) & Подстановка: A = 3, B = 2, C = 1, Max = Max & Прямой ход \\
      & & maxWoutCut(3, 2, 1, Max) & \\
      & & maxWoutCut(A, B, C, Max) & \\
    \hline
    2 & A >= B & Проверка: 3 >= 2 & Прямой ход \\
      & A >= C & & \\
      & Max = A & & \\
    \hline
    3 & A >= C & Проверка: 3 >= 1 & Прямой ход \\
      & Max = A & & \\
    \hline
    4 & Пусто & Подстановка: Max = 3 & Прямой ход \\
    \hline
    5 & Пусто & \textbf{Результат:} Max = 3 & Обратный ход \\
    \hline
    6 & maxWoutCut(3, 2, 1, Max) & Подстановка: A = 3, B = 2, C = 1, Max = Max & Прямой ход \\
      & & maxWoutCut(3, 2, 1, Max) & \\
      & & maxWoutCut(A, B, C, Max) & \\
    \hline
    7 & B > A & Проверка: 2 > 3 & Обратный ход \\
      & B >= C & & \\
      & Max = B & & \\
    \hline
    8 & maxWoutCut(3, 2, 1, Max) & Подстановка: A = 3, B = 2, C = 1, Max = Max & Прямой ход \\
      & & maxWoutCut(3, 2, 1, Max) & \\
      & & maxWoutCut(A, B, C, Max) & \\
    \hline
    9 & C > A & Проверка: 1 > 3 & Обратный ход \\
      & C > B & & \\
      & Max = C & & \\
    \hline
\end{longtable}
}

\section{ВЫВОДЫ}

Эффективность программы может быть достигнута за счет использования отсечения (!), которое останавливает поиск правил и фактов в программе.
