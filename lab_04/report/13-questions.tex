\section{ВОПРОСЫ}

\begin{enumerate}
    \item \textbf{В какой части правила сформулировано знание? Это знание о чем, с формальной точки зрения?}

        Знания о предметной области выражаются на языке Пролог в виде предложений, называемых утверждениями (clauses). Каждое утверждение заканчивается точкой и описывает какое-либо отношение, свойство, объект или закономерность. Структура утверждения проста и имеет одну из форм:
        \begin{itemize}
            \item <заголовок>. -- факт
            \item <заголовок>:- <тело>. -- правило, где заголовок является предикатом и полностью характеризует описы- ваемое отношение.
        \end{itemize}

    \item \textbf{Что такое процедура?}

        В Prolog существует понятие процедуры. Процедурой называется совокупность правил, заголовки которых имеют одно и то же имя и одну и ту же арность (местность), т.е. это совокупность правил, описывающих одно определенное отношение. Отношение, определяемое процедурой, называется предикатом.

    \item \textbf{Сколько в БЗ текущего задания процедур?}

        В текущем задании четыре процедуры.

    \item \textbf{Что такое пример терма, это частный случай терма, пример? Как строится пример? }

        Терм $B$ называется \textbf{примером} терма $A$, если существует такая подстановка $\Theta$, что $B = A\Theta$. Пример терма строится во время поиска решений при подстановке. Для построения примеров, система связывает переменные с конкретными термами, все примеры для текущего вопроса хранятся в стеке.

    \item \textbf{Что такое наиболее общий пример?}

        Терм $C$ называется \textbf{общим примером} термов $A$ и $B$, если существуют такие подстановки $\Theta_1$ и $\Theta_2$, что $C = A\Theta_1$ и $C=B\Theta_2$.

    \item \textbf{Назначение и результат работы алгоритма унификации. Что значит двунаправленная передача параметров при работе алгоритма унификации, поясните на примере одного из случаев пункта  3.}

        Поэтому для выполнения логического вывода используется механизм (алгоритм) унификации, встроенный в систему. Унификация – операция, которая позволяет формализовать процесс логического вывода (наряду с правилом резолюции). С практической точки зрения - это основной вычислительный шаг, с помощью которого происходит:
        \begin{itemize}
            \item Двунаправленная передача параметров процедурам,
            \item Неразрушающее присваивание,
            \item Проверка условий (доказательство).
        \end{itemize}
        В процессе работы система выполняет большое число унификаций.

        Таким образом, с помощью алгоритма унификации происходит двунаправленная передача параметров процедурам. Например, из внешнего мира в программу для дальнейшего использования или из программы во внешний мир – значения интересующего нас параметра.

    \item \textbf{В каком случае запускается механизм отката?}

        В том месте программы, где возможен выбор нескольких вариантов, Пролог сохраняет в специальный стек точку возврата для последующего возвращения в эту позицию. Точка возврата содержит информацию, необходимую для возобновления процедуры при откате. Выбирается один из возможных вариантов, после чего продолжается выполнение программы.

Во всех точках программы, где существуют альтернативы, в стек заносятся указатели. Если впоследствии окажется, что выбранный вариант не приводит к успеху, то осуществляется откат к последней из имеющихся в стеке точек программы, где был выбран один из альтернативных вариантов. Выбирается очередной вариант, программа продолжает свою работу. Если все варианты в точке уже были использованы, то регистрируется неудачное завершение и осуществляется переход на предыдущую точку возврата, если такая есть. При откате все связанные переменные, которые были означены после этой точки, опять освобождаются.

    \item \textbf{Виды и назначение переменных в Prolog. Примеры из задания.  Почему использованы те или другие переменные (примеры из задания)?}

    \begin{itemize}
        \item Именованная -- обозначается комбинацией символов латинского алфавита, цифр и символа подчеркивания, начинающейся с прописной буквы или символа подчеркивания (X, Student, \_X)
        \item Анонимная -- обозначается символом почеркивания (\_)
    \end{itemize}

    Переменные в момент фиксации утверждений в программе, обозначая некоторый неизвестный объект из некоторого множества объектов, не имеют значения. Значения для переменных могут быть установлены Prolog-системой только в процессе поиска ответа на вопрос, то есть реализации программы.

    Переменные предназначены для передачи значений <<во времени и пространстве>>. Переменные в факты и правила входят только с квантором всеобщности. А в вопросы переменные входят только с квантором существования.

    В процессе выполнения программы переменные могут сязываться с различными объектами -- \textbf{конкретизироваться}. Это относится только к именованным переменным. Анонимные переменные не могут быть связаны со значением.

    В более общей -- абстрактной форме смормулировано предложение содержащее переменные.

    Например, на строчке 45: car(Lastname, Mark, Color, \_, City), используется 5 переменных, Mark и Color используются для унификации, Lastname и City используются для вывода найденного результата, а оставшийся аргумент не нужно никак использовать, поэтому используется анонимная переменная (\_).
\end{enumerate}
